% Options for packages loaded elsewhere
\PassOptionsToPackage{unicode}{hyperref}
\PassOptionsToPackage{hyphens}{url}
%
\documentclass[
  ignorenonframetext,
]{beamer}
\usepackage{pgfpages}
\setbeamertemplate{caption}[numbered]
\setbeamertemplate{caption label separator}{: }
\setbeamercolor{caption name}{fg=normal text.fg}
\beamertemplatenavigationsymbolsempty
% Prevent slide breaks in the middle of a paragraph
\widowpenalties 1 10000
\raggedbottom
\setbeamertemplate{part page}{
  \centering
  \begin{beamercolorbox}[sep=16pt,center]{part title}
    \usebeamerfont{part title}\insertpart\par
  \end{beamercolorbox}
}
\setbeamertemplate{section page}{
  \centering
  \begin{beamercolorbox}[sep=12pt,center]{part title}
    \usebeamerfont{section title}\insertsection\par
  \end{beamercolorbox}
}
\setbeamertemplate{subsection page}{
  \centering
  \begin{beamercolorbox}[sep=8pt,center]{part title}
    \usebeamerfont{subsection title}\insertsubsection\par
  \end{beamercolorbox}
}
\AtBeginPart{
  \frame{\partpage}
}
\AtBeginSection{
  \ifbibliography
  \else
    \frame{\sectionpage}
  \fi
}
\AtBeginSubsection{
  \frame{\subsectionpage}
}
\usepackage{amsmath,amssymb}
\usepackage{lmodern}
\usepackage{ifxetex,ifluatex}
\ifnum 0\ifxetex 1\fi\ifluatex 1\fi=0 % if pdftex
  \usepackage[T1]{fontenc}
  \usepackage[utf8]{inputenc}
  \usepackage{textcomp} % provide euro and other symbols
\else % if luatex or xetex
  \usepackage{unicode-math}
  \defaultfontfeatures{Scale=MatchLowercase}
  \defaultfontfeatures[\rmfamily]{Ligatures=TeX,Scale=1}
\fi
\usetheme[]{CambridgeUS}
\usecolortheme{rose}
\usefonttheme{structurebold}
% Use upquote if available, for straight quotes in verbatim environments
\IfFileExists{upquote.sty}{\usepackage{upquote}}{}
\IfFileExists{microtype.sty}{% use microtype if available
  \usepackage[]{microtype}
  \UseMicrotypeSet[protrusion]{basicmath} % disable protrusion for tt fonts
}{}
\makeatletter
\@ifundefined{KOMAClassName}{% if non-KOMA class
  \IfFileExists{parskip.sty}{%
    \usepackage{parskip}
  }{% else
    \setlength{\parindent}{0pt}
    \setlength{\parskip}{6pt plus 2pt minus 1pt}}
}{% if KOMA class
  \KOMAoptions{parskip=half}}
\makeatother
\usepackage{xcolor}
\IfFileExists{xurl.sty}{\usepackage{xurl}}{} % add URL line breaks if available
\IfFileExists{bookmark.sty}{\usepackage{bookmark}}{\usepackage{hyperref}}
\hypersetup{
  pdftitle={08\_gaussian\_ch8},
  pdfauthor={Randy},
  hidelinks,
  pdfcreator={LaTeX via pandoc}}
\urlstyle{same} % disable monospaced font for URLs
\newif\ifbibliography
\usepackage{graphicx}
\makeatletter
\def\maxwidth{\ifdim\Gin@nat@width>\linewidth\linewidth\else\Gin@nat@width\fi}
\def\maxheight{\ifdim\Gin@nat@height>\textheight\textheight\else\Gin@nat@height\fi}
\makeatother
% Scale images if necessary, so that they will not overflow the page
% margins by default, and it is still possible to overwrite the defaults
% using explicit options in \includegraphics[width, height, ...]{}
\setkeys{Gin}{width=\maxwidth,height=\maxheight,keepaspectratio}
% Set default figure placement to htbp
\makeatletter
\def\fps@figure{htbp}
\makeatother
\setlength{\emergencystretch}{3em} % prevent overfull lines
\providecommand{\tightlist}{%
  \setlength{\itemsep}{0pt}\setlength{\parskip}{0pt}}
\setcounter{secnumdepth}{-\maxdimen} % remove section numbering
\AtBeginSubsection{}
\AtBeginSection{}
\ifluatex
  \usepackage{selnolig}  % disable illegal ligatures
\fi

\title{08\_gaussian\_ch8}
\author{Randy}
\date{9/23/2021}

\begin{document}
\frame{\titlepage}

\begin{frame}[allowframebreaks]
  \tableofcontents[hideallsubsections]
\end{frame}
\hypertarget{chapter-8-approximation-methods-for-large-datasets}{%
\section{Chapter 8 Approximation Methods for Large
Datasets}\label{chapter-8-approximation-methods-for-large-datasets}}

\begin{frame}{\(\mathcal O(.)\) problem}
\protect\hypertarget{mathcal-o.-problem}{}
\begin{itemize}
\item
  A significant problem with Gaussian process prediction is that it
  typically scales as \(\mathcal O(n^3)\).
\item
  prohibitive problems for large dataset (e.g.~\(n > 10, 000\)):

  \begin{itemize}
  \tightlist
  \item
    storing the Gram matrix
  \item
    solving the associated linear systems
  \end{itemize}
\end{itemize}
\end{frame}

\hypertarget{reduced-rank-approximations-of-the-gram-matrix}{%
\section{8.1 Reduced-rank Approximations of the Gram
Matrix}\label{reduced-rank-approximations-of-the-gram-matrix}}

\begin{frame}{Inversion Lemma}
\protect\hypertarget{inversion-lemma}{}
To invert the matrix \(K + \sigma_n^2 I\) (or at least to solve a linear
system (\(K + \sigma_n^2 I)\pmb v = \pmb y\) for \(\pmb v\))

\begin{itemize}
\item
  \(K\) has rank \(q\) (so that it can be represented in the form
  \(K = QQ^{\top}\);
\item
  where \(Q\) is an \(n \times q\) matrix)
\item
  Matrix inversion can be speeded up using the matrix inversion lemma
  eq. (A.9)
\end{itemize}

\[
(Z + UWV^{\top})^{-1} = Z^{-1} - Z^{-1}U(W^{-1} + V^{\top}Z^{-1}U)^{-1}V^{\top}Z^{-1} \ \ \ \ (A.9)
\]

\begin{itemize}
\tightlist
\item
  Result as
  \((QQ^{\top} + \sigma_n^2I_n)^{-1} = \sigma_n^{-2} I_n - \sigma_n^{-2} Q(\sigma_n^2 I_q + Q^{\top} Q)^{-1}Q^{\top}\).
\end{itemize}

\alert{Notice that the inversion of an $n \times n$ matrix has now been transformed to the inversion of a $q \times q$ matrix.}
\end{frame}

\begin{frame}{Inversion Lemma for the kernel with N features}
\protect\hypertarget{inversion-lemma-for-the-kernel-with-n-features}{}
\begin{itemize}
\item
  The Gram matrix will have rank \(min(n, N)\) so that exploitation of
  this structure will be beneficial if \(n > N\).
\item
  Even if the kernel is non-degenerate it may happen that it has a
  fast-decaying eigen-spectrum
\item
  so that a reduced-rank approximation will be accurate
\end{itemize}
\end{frame}

\begin{frame}{Even K is not of \(rank < n\)}
\protect\hypertarget{even-k-is-not-of-rank-n}{}
\begin{itemize}
\item
  still consider reduced-rank approximations to \(K\)
\item
  with the optimal reduced-rank approximation of \(K\) w.r.t. the
  \textbf{Frobenius norm} (see eq. (A.16))
\end{itemize}

\[
\| A\|_F^2 = \sum_{i=1}^{n_1} \sum_{j=1} ^{n_2} |a_{ij}|^2 = tr(AA^{\top}) \ \ \ \ \ (A.16)
\]

\begin{itemize}
\item
  \(U_q {\Lambda}_q U_q^{\top}\) with \({\Lambda}_q\) is the diagonal
  matrix with the first \(q\) eigenvalues of \(K\) and \(U_q\) is the
  matrix of the corresponding orthonormal eigenvectors
\item
  Limit of computing the eigen-decomposition is an \(\mathcal O(n^3)\)
  operation
\item
  However, it does suggest that if we can more cheaply obtain an
  approximate eigen-decomposition (may give rise to a reduced rank
  approximation)
\end{itemize}
\end{frame}

\begin{frame}{Setting up an \textbf{active set}}
\protect\hypertarget{setting-up-an-active-set}{}
A subset \(I\) of the original \(n\) data points, called the
\textbf{active set}.

\begin{itemize}
\tightlist
\item
  Setting \(I\) as size \(m < n\) (\(I\) is for the included data point)
\item
  Remaining \(n - m\) data points form the set \(R\) (\(R\) is for the
  remaining points)
\item
  WOLOG: the data points are ordered so that set \(I\) comes first
\item
  \(K\) can be partitioned as
\end{itemize}

\[
K =  
\begin{pmatrix}
K_{mm} & K_{m(n-m)} \\
K_{(n-m)m} &  K_{(n-m)(n-m)}  
\end{pmatrix} 
\ \ \ \ \  (8.1)
\]
\end{frame}

\begin{frame}{To approximate the eigenfunctions of a kernel using the
Nystrom method.}
\protect\hypertarget{to-approximate-the-eigenfunctions-of-a-kernel-using-the-nystrom-method.}{}
\begin{itemize}
\item
  Compute the eigenvectors and eigenvalues of \(K_{mm}\) and denote them
  \(\{{\lambda}_i^{(m)}\}^m_{i=1}\) and \(\{\pmb u_i^{(m)}\}^m_{i=1}\).
\item
  Extended to all \(n\) points using \textbf{eq. (4.44)}
  (\(\pmb k(\pmb x') = \{_C\ k(\pmb x_i,\ \pmb x') \}_{i=1}^n\))
\end{itemize}

\[
\phi_i(\pmb x') \simeq \frac {\sqrt n } {\lambda_i^{mat}} \pmb k(\pmb x')^{\top}\pmb u_i \ \ \ \ \ (4.44)
\]

\begin{itemize}
\tightlist
\item
  \({\tilde \lambda}^{(n)}_i \stackrel {\bigtriangleup} = \frac n m {\lambda}^{(m)}_i,\  i = 1, . . . , m \ \ \ \ \ (8.2)\)
\item
  \(\pmb {\tilde u}^{(n)}_i \stackrel {\bigtriangleup} = \sqrt {\frac m n} \frac 1 {{\lambda}^{(m)}_i} K_{nm} \pmb u_i^{(m)},\  i = 1, . . . , m \ \ \ \ (8.3)\)
\item
  with the scaling of \(\pmb {\tilde u}_i^{(n)}\) has been chosen so
  that \(|\pmb {\tilde u}_i^{(n)}| \simeq 1\)
\end{itemize}
\end{frame}

\begin{frame}{Nystrom approximation to K}
\protect\hypertarget{nystrom-approximation-to-k}{}
In general we can choice the approximate eigenvalues/vectors to include
in approximation of \(K\)

\begin{itemize}
\item
  Choosing the first \(p\) values:
  \({ \tilde K} = \sum^p_{i=1} {\tilde \lambda}_i^{(n)} \pmb {\tilde u}^{(n)}_i (\pmb {\tilde u}^{(n)}_i)^{\top}\)
\item
  Now set \(p = m\) to obtain
  \({ \tilde K} = K_{nm} K_{mm}^{-1} K_{mn} \ \ \ \  (8.4)\)
\item
  Combining equations 8.2, 8.3, and 8.4
\end{itemize}
\end{frame}

\begin{frame}{}
\protect\hypertarget{section}{}
\begin{block}{Computation of \({ \tilde K}\) takes time
\(\mathcal O(m^2n)\)}
\protect\hypertarget{computation-of-tilde-k-takes-time-mathcal-om2n}{}
\begin{itemize}
\tightlist
\item
  The eigen-decomposition of \(K_{mm}\) is \(\mathcal O(m^3)\)
\item
  The computation of each \(\pmb {\tilde u}^{(n)}_i\) is
  \(\mathcal O(mn)\)
\item
  Up to (\(10^6 \times 10^6\) in size by \textbf{Fowlkes et
  al.~{[}2001{]}}).
\end{itemize}
\end{block}
\end{frame}

\begin{frame}{}
\protect\hypertarget{section-1}{}
The Nystrom approximation has been applied above to approximate the
elements of \(K\).

However, using the approximation for the \(i\)th eigen-function

\(\tilde \phi_i(\pmb x) = (\sqrt m / {\lambda}_i^{(m)}) \pmb k_m(\pmb x)^{\top} \pmb u_i^{(m)}\)
with (\(\pmb k(\pmb x') = \{_C\ k(\pmb x_i,\ \pmb x') \}_{i=1}^n\))

\begin{itemize}
\tightlist
\item
  \({\lambda}_i' \simeq {\lambda}_i^{(m)}/m\) it is easy to see that in
  general we obtain an approximation for the kernel
  \(\pmb k(\pmb x, \pmb x') = \sum^N_{i=1} {\lambda}_i \phi_i(\pmb x) \phi_i(\pmb x')\)
  as
\end{itemize}

\[
\begin{split}
{\tilde k}(\pmb x, \pmb x') & = \sum^m_{i=1} \frac {{\lambda}^{(m)}_i} m \tilde \phi_i(\pmb x) \tilde \phi_i(\pmb x')  \ \ \ \ \ (8.5) \\
& = \sum^m_{i=1} \frac {{\lambda}^{(m)}_i} m 
\frac m {( {\lambda}_i^{(m)})^2} \pmb k_m(\pmb x) ^{\top} \pmb u_i^{(m)} (\pmb u_i^{(m)})^{\top} \pmb k_m(\pmb x') \ \ \ \ \ (8.6)\\
& = \pmb k_m(\pmb x)^{\top} K_{mm}^{-1} \pmb k_m(\pmb x') \ \ \ \ \ (8.7)
\end{split}
\]
\end{frame}

\begin{frame}{}
\protect\hypertarget{section-2}{}
\begin{itemize}
\item
  By multiplying out eq. (8.4) using \(K_{mn} = [K_{mm}K_{m(n-m)}]\) it
  is easy to show that \(K_{mm} = { \tilde K}_{mm}\),
  \(K_{m(n-m)} = { \tilde K}_{m(n-m)}\),
  \(K_{(n-m)m} = { \tilde K}_{(n-m)m}\), but that
  \({\tilde K}_{(n-m)(n-m)} = K_{(n-m)m}K_{mm}^{-1} K_{m(n-m)}\).
\item
  The difference \(K_{(n-m)(n-m)} - {\tilde K}_{(n-m)(n-m)}\) is in fact
  the \textbf{Schur complement} of \(K_mm\) {[}Golub and Van Loan, 1989,
  p.~103{]}.
\item
  \(K_{(n-m)(n-m)} - {\tilde K}_{(n-m)(n-m)}\) is positive
  semi-definite;
\item
  If a vector \(\pmb f\) is partitioned as
  \(\pmb f^{\top} = (\pmb f_m^{\top},\ \pmb f_{n-m}^{\top})\) and
  \(\pmb f\) has a Gaussian distribution with zero mean and covariance
  \(K\) then \(\pmb f_{n-m} | \pmb f_m\) has the \textbf{Schur
  complement} as its covariance matrix, see eq. (A.6).
\end{itemize}
\end{frame}

\begin{frame}{}
\protect\hypertarget{section-3}{}
\[
\begin{split}
& \begin{bmatrix}
\pmb x\\
\pmb y
\end{bmatrix}
\sim
\mathcal N 
\begin{pmatrix}
\begin{bmatrix}
\pmb \mu_x\\
\pmb \mu_y
\end{bmatrix},
\begin{bmatrix}
A & C\\
C^{\top} & B
\end{bmatrix}
\end{pmatrix} = 
\mathcal N 
\begin{pmatrix}
\begin{bmatrix}
\pmb \mu_x\\
\pmb \mu_y
\end{bmatrix},
\begin{bmatrix}
\tilde A & \tilde C\\
\tilde C^{\top} & \tilde B
\end{bmatrix}^{-1}
\end{pmatrix} \ \ \ \ \ (A.5)\\
& \pmb x \sim \mathcal N (\pmb \mu_x, A)\\
& \pmb x|\pmb y \sim \mathcal N (\mu_x + CB^{−1}(\pmb y − \pmb \mu_y),\  A − CB^{−1}C^{\top}\ \ \ \ \ \ (A.6a)\\
& \pmb x|\pmb y \sim \mathcal N (\mu_x − \tilde A^{−1} \tilde C(\pmb y − \pmb \mu_y),\  \tilde A^{−1})\ \ \ \ \ \ (A.6b)
\end{split}
\]
\end{frame}

\begin{frame}{An alternative view}
\protect\hypertarget{an-alternative-view}{}
\begin{itemize}
\item
  \textbf{Nystrom approximation} was derived in the above fashion by
  \textbf{Williams and Seeger {[}2001{]}} for application to kernel
  machines
\item
  The same approximation is due to \textbf{Smola and Sch¨olkopf
  {[}2000{]}}
\item
  To approximate the kernel centered on point \(\pmb x_i\) as a linear
  combination of kernels from the active set
\end{itemize}

\[
\pmb k(\pmb x_i, \pmb x) \simeq \sum_{j \in I} c_{ij} k(\pmb x_j, \pmb x) \stackrel {\bigtriangleup} = \hat k (\pmb x_i, \pmb x) \ \ \ \ \ (8.8)
\]

for some coefficients \(\{c_{ij}\}\) that are to be determined so as to
optimize the approximation
\end{frame}

\begin{frame}{A reasonable criterion to minimize}
\protect\hypertarget{a-reasonable-criterion-to-minimize}{}
\[ 
\begin{split}
E(C) & = \sum^n_{i=1} \|k(\pmb x_i, \pmb x) - \hat k(\pmb x_i, \pmb x)\|^2_{\mathcal H} \ \ \ \ \ (8.9) \\
& = tr\ K - 2 tr(CK_{mn}) + tr(CK_{mm}C^{\top}) \ \ \ \ \ (8.10)
\end{split}
\]

\begin{itemize}
\item
  The coefficients are arranged into a \(n \times m\) matrix \(C\).
\item
  Minimizing \(E(C)\) w.r.t. \(C\) gives \(C_{opt} = K_{nm}K_{mm}^{-1}\)
\item
  Thus we obtain the approximation \(\hat K = K_{nm}K_{mm}^{-1} K_{mn}\)
  in agreement with eq. (8.4).
\item
  \(E(C_{opt}) = tr(K - \hat K)\)
\end{itemize}
\end{frame}

\begin{frame}{}
\protect\hypertarget{section-4}{}
\textbf{Smola and Scholkopf {[}2000{]}} suggest a \textbf{greedy
algorithm} to choose points to include into the \textbf{active set} so
as to minimize the error criterion.

\begin{itemize}
\tightlist
\item
  \(\mathcal O(mn)\) operations to evaluate the change in \(E\) due to
  including one new datapoint
\item
  it is infeasible to consider all members of set \(R\) for inclusion on
  each iteration
\item
  instead \textbf{Smola and Scholkopf {[}2000{]}} suggest finding the
  best point to include from a randomly chosen subset of set \(R\) on
  each iteration.
\end{itemize}
\end{frame}

\begin{frame}{}
\protect\hypertarget{section-5}{}
\textbf{Drineas and Mahoney {[}2005{]}} used biased sampling with
replacement

\begin{itemize}
\tightlist
\item
  choosing column \(i\) of \(K\) with probability \(\propto k_{ii}^2\)
\item
  a pseudoinverse of the inner \(m \times m\) matrix
\end{itemize}

To provide probabilistic bounds on the quality of the approximation
\end{frame}

\begin{frame}{}
\protect\hypertarget{section-6}{}
\textbf{Frieze et al.~{[}1998{]}} had developed an approximation to the
singular value decomposition (SVD) of a rectangular matrix

\begin{itemize}
\item
  using a weighted random subsampling of its rows and columns, and
  probabilistic error bounds.
\item
  However, this is rather different from the Nystrom approximation
\end{itemize}

** Fine and Scheinberg {[}2002{]}** suggest an alternative low-rank
approximation to \(K\) using the incomplete Cholesky factorization

\begin{itemize}
\tightlist
\item
  when computing the Cholesky decomposition of \(K\) pivots below a
  certain threshold are skipped.
\item
  If the number of pivots greater than the threshold is \(k\) the
  incomplete Cholesky factorization takes time \(\mathcal O(nk^2)\)
\end{itemize}
\end{frame}

\hypertarget{greedy-approximation}{%
\section{8.2 Greedy Approximation}\label{greedy-approximation}}

\begin{frame}{8.2 Greedy Approximation}
\protect\hypertarget{greedy-approximation-1}{}
\begin{itemize}
\item
  an active set of training points of size \(m\) selected from the
  training set of size \(n > m\)
\item
  assume that it is impossible to search for the optimal subset of size
  \(m\) due to combinatorics.
\item
  The points in the active set could be selected randomly
\item
  but if the points are selected greedily w.r.t. some criterion, the
  results are better.
\item
  greedy approaches are also known as forward selection strategies.
\end{itemize}
\end{frame}

\begin{frame}{Algorithm}
\protect\hypertarget{algorithm}{}
\includegraphics{figure/8-1.png}
\end{frame}

\begin{frame}{}
\protect\hypertarget{section-7}{}
This is achieved by evaluating some criterion \(\Delta\) and selecting
the data point that optimizes this criterion.

For some algorithms it can be too expensive to evaluate \(\Delta\) on
all points in \(R\), so some working set \(J \subset R\) can be chosen
instead, usually at random from \(R\).

Greedy selection is used with the \textbf{subset of regressors (SR)},
\textbf{subset of datapoints (SD)} and the \textbf{projected process
(PP)} methods
\end{frame}

\hypertarget{approximations-for-gpr-with-fixed-hyperparameters}{%
\section{8.3 Approximations for GPR with Fixed
Hyperparameters}\label{approximations-for-gpr-with-fixed-hyperparameters}}

\begin{frame}{Total six approximation schemes for GPR below:}
\protect\hypertarget{total-six-approximation-schemes-for-gpr-below}{}
\begin{itemize}
\item
  The subset of regressors (SR)
\item
  The Nystrom method
\item
  The subset of datapoints (SD)
\item
  The projected process (PP) approximation
\item
  The Bayesian committee machine (BCM)
\item
  The iterative solution of linear systems
\end{itemize}
\end{frame}

\begin{frame}{8.3.1 Subset of Regressors}
\protect\hypertarget{subset-of-regressors}{}
\textbf{Silverman {[}1985, sec.~6.1{]}} showed:

\begin{itemize}
\item
  the mean GP predictor can be obtained from a finite-dimensional
  generalized linear regression model
\item
  \(f(\pmb x_*) = \sum^n_{i=1} \alpha_i k(\pmb x_*,\ \pmb xi)\) with a
  prior \(\pmb \alpha \sim \mathcal N (\pmb 0,\ K^{-1})\)
\item
  the mean prediction for linear regression model in feature space given
  by eq. (2.11),
\end{itemize}

\[
\begin{split}
& \bar f(\pmb x_*) =  \sigma_n^{-2} \phi(\pmb x_*)^{\top} A^{-1}\Phi y\\
& A =  \Sigma_p^{- 1} +  \sigma_n^{-2}\Phi \Phi^{\top} \\
& \phi(\pmb x_*) = \pmb k(x_*)\\
& \Phi = \Phi^{\top} = K\\
& \Sigma_p^{-1} = K\\ 
\end{split}
\]
\end{frame}

\begin{frame}{}
\protect\hypertarget{section-8}{}
\[
\begin{split}
\bar f(\pmb x_*) & =  \sigma_n^{-2} \pmb k^{\top}(\pmb x_*)[ \sigma_n^{-2}K(K +  \sigma_n^2I)]^{-1}K \pmb y \ \ \ \ (8.11)\\
& = \pmb k^{\top}(\pmb x_*)(K +  \sigma_n^2I)^{-1} \pmb y \ \ \ \ \ (8.12)
\end{split}
\]

\begin{itemize}
\item
  this result is in agreement with eq. (2.25)
\item
  however, that the predictive (co)variance of this model is different
  from full GPR.
\end{itemize}
\end{frame}

\begin{frame}{}
\protect\hypertarget{section-9}{}
A simple approximation to this model is to consider only a subset of
regressors, so that

\[
f_{SR}(\pmb x_*) = \sum^m_{i=1} \alpha_ik(\pmb x_* , \pmb x_i),\  \alpha_m \sim \mathcal N(\pmb 0, K_{mm}^{−1}) \ \ \ \ (8.13)
\]

by
\(f_*|\pmb x_*, X, \pmb y \sim \mathcal N (\frac {1}{\sigma^2_n} \phi(\pmb x_*)^{\top}A^{-1}\Phi\pmb y,\ \phi(\pmb x_*)T{\top}A^{-1}\phi(\pmb x_*)) \ \ \ \ (2.11)\)

\[
\begin{split}
\bar f_{SR} (\pmb x_*) = \pmb k_m(\pmb x_*)^{\top} (K_{mn}K_{nm} +  \sigma_n^2K_{mm})^{-1}K_{mn} \pmb y \ \ \ \ (8.14)\\
\mathbb V[f_{SR}(\pmb x_*)] =  \sigma_n^2 \pmb k_m(\pmb x_*)^{\top}(K_{mn}K_{nm} +  \sigma_n^2K_{mm})^{-1}\pmb k_m(\pmb x_*) \ \ \ \ (8.15)\\
\bar \alpha_m = (K_{mn}K_{nm} +  \sigma_n^2K_{mm})^{-1}K_{mn} \pmb y \ \ \ (8.16)
\end{split}
\]
\end{frame}

\begin{frame}{\textbf{``subset of regressors'' (SR)} was suggested to us
by \textbf{G. Wahba}.}
\protect\hypertarget{subset-of-regressors-sr-was-suggested-to-us-by-g.-wahba.}{}
\begin{itemize}
\item
  The computations for equations 8.14 and 8.15 take time
  \(\mathcal O(m^2n)\) to carry out the necessary matrix computations.
\item
  After this the prediction of the mean for a new test point takes time
  \(\mathcal O(m)\), and the predictive variance takes
  \(\mathcal O(m^2)\).
\end{itemize}

Under the subset of regressors model we have
\(f \sim \mathcal N(0, {\tilde K})\) where \({\tilde K}\) is defined as
in eq. (8.4).

Thus the log marginal likelihood under this model is
\(\log p_{SR}(y|X) = -{\frac 1 2} \log | {\tilde K} + \sigma_n^2 I_n| - {\frac 1 2}\ pmb y^{\top} ({\tilde K} + \sigma_n^2 I_n)^{-1} \pmb y - {\frac n 2} \log(2\pi) \ \ \  (8.17)\)

Notice that the covariance function defined by the SR model has the form
\({\tilde K}(\pmb x, \pmb x') = \pmb k(\pmb x)^{\top} K_{mm}^{-1} \pmb k(\pmb x')\),
which is exactly the same as that from the Nystrom approximation for the
covariance function eq. (8.7).
\end{frame}

\begin{frame}{}
\protect\hypertarget{section-10}{}
In fact if the covariance function \(k(\pmb x, \pmb x')\) in the
predictive mean and variance equations 2.25 and 2.26 is replaced
systematically with \({\tilde K}(\pmb x, \pmb x')\) we obtain equations
8.14 and 8.15, as shown in Appendix 8.6.

If the kernel function decays to zero for \(|x| \rightarrow \infty\) for
fixed \(\pmb x'\), then \({\tilde K}(\pmb x, \pmb x)\) will be near zero
when \(\pmb x\) is distant from points in the set \(I\).

This will be the case even when the kernel is stationary so that
\(k(x, x)\) is independent of \(x\).

Thus we might expect that using the approximate kernel will give poor
predictions, especially underestimates of the predictive variance, when
\(\pmb x\) is far from points in the set \(I\).
\end{frame}

\begin{frame}{}
\protect\hypertarget{section-11}{}
An interesting idea suggested by \textbf{Rasmussen and Quinonero-Candela
{[}2005{]}} to mitigate this problem

\begin{itemize}
\item
  to define the SR model with \(m + 1\) basis functions, where the extra
  basis function is centered on the test point \(x_*\)
\item
  so that
  \(y_{SR*} (\pmb x_*) = \sum^m_{i=1} \alpha_i k(\pmb x_* , \pmb x_i) + \alpha_* k(\pmb x_* , \pmb x_*)\).
\item
  This model can then be used to make predictions, and it can be
  implemented efficiently using the partitioned matrix inverse equations
  A.11 and A.12.
\end{itemize}
\end{frame}

\begin{frame}{}
\protect\hypertarget{section-12}{}
The effect of the extra basis function centered on \(\pmb x_*\) is to
maintain predictive variance at the test point.

\begin{itemize}
\item
  One simple method is to choose subset \(I\) randomly from \(X\)
\item
  Another is to run clustering on \({\pmb x_i}^n_{i=1}\) to obtain
  centers.
\item
  Alternatively, a number of greedy forward selection algorithms for
  \(I\) have been proposed:

  \begin{itemize}
  \item
    \textbf{Luo and Wahba {[}1997{]}} choose the next kernel so as to
    minimize the residual sum of squares (RSS)
    \(|y -K_{nm} \pmb \alpha_m|^2\) after optimizing \(\pmb \alpha_m\)
  \item
    \textbf{Smola and Bartlett {[}2001{]}} choose as their criterion the
    quadratic form
  \end{itemize}
\end{itemize}

\[
{\frac 1 {2 \sigma_n^2}} |\pmb y - K_{nm} \pmb {\bar \alpha}_m|^2 + 
\pmb {\bar \alpha}_m ^{\top} K_{mm} \pmb {\bar \alpha}_m  =
\pmb y^{\top} ({\tilde K} +  \sigma_n^2I_n)^{-1} \pmb y \ \ \ \ (8.18)
\]
\end{frame}

\begin{frame}{}
\protect\hypertarget{section-13}{}
\begin{itemize}
\item
  Alternatively, \textbf{Quinonero-Candela {[}2004{]}} suggests using
  the approximate \(\log p_{SR}(y|X)\) (see eq. (8.17)) as the selection
  criterion.
\item
  the quadratic term from eq. (8.18) is one of the terms comprising
  \(\log p_{SR}(y|X)\).
\item
  For all these suggestions the complexity of evaluating the criterion
  on a new example is \(\mathcal O(mn)\), by making use of partitioned
  matrix equations.
\item
  Thus it is likely to be too expensive to consider all points in R on
  each iteration
\item
  Note that the SR model is obtained by selecting some subset of the
  data points of size \(m\) in a random or greedy manner.
\item
  The relevance vector machine (RVM) described in section 6.6 has a
  similar flavour
\item
  it automatically selects (in a greedy fashion) which data points to
  use in its expansion.
\item
  However, note one important difference which is that the RVM uses a
  diagonal prior on the \(\alpha\)'s, while for the SR method we have
  \(\alpha_m \sim \mathcal N (0,\  K_{mm}^{-1})\).
\end{itemize}
\end{frame}

\begin{frame}{8.3.2 The Nystrom Method for approximate GPR}
\protect\hypertarget{the-nystrom-method-for-approximate-gpr}{}
\textbf{Williams and Seeger {[}2001{]}} suggested:

\begin{itemize}
\item
  approximating the GPR equations by replacing the matrix \(K\) by
  \({\tilde K}\) in the mean and variance prediction equations 2.25 and
  2.26.
\item
  in this proposal the covariance function \(k\) is not systematically
  replaced by \({\tilde k}\)
\item
  it is only occurrences of the matrix \(K\) that are replaced.
\item
  As for the SR model the time complexity is \(\mathcal O(m^2n)\) to
  carry out the necessary matrix computations
\item
  then \(\mathcal O(n)\) for the predictive mean of a test point
\item
  \(\mathcal O(mn)\) for the predictive variance
\end{itemize}
\end{frame}

\begin{frame}{}
\protect\hypertarget{section-14}{}
Experimental evidence in \textbf{Williams et al.~{[}2002{]}} suggests:

\begin{itemize}
\item
  for large m the SR and Nystrom methods have similar performance
\item
  but for small \(m\) the Nystrom method can be quite poor
\item
  Also the fact that \(k\) is not systematically replaced by
  \({\tilde K}\) means that the approximated predictive variance might
  be negative.
\item
  For these reasons, we do not recommend the Nystrom method over the SR
  method.
\item
  However, the Nystrom method can be effective when \({\lambda}_{m+1}\),
  the \((m + 1)\)th eigenvalue of \(K\), is much smaller than
  \(\sigma_n\).
\end{itemize}
\end{frame}

\begin{frame}{8.3.3 Subset of Datapoints}
\protect\hypertarget{subset-of-datapoints}{}
\begin{itemize}
\item
  to keep the GP predictor, but only on a smaller subset of size \(m\)
  of the data.
\item
  Although this is clearly wasteful of data, it can make sense if the
  predictions obtained with \(m\) points are sufficiently accurate for
  our needs.
\item
  it can make sense to select which points are taken into the active set
  \(I\), and typically this is achieved by greedy algorithms.
\item
  However, one has to be wary of the amount of computation that is
  needed, if one considers each member of \(R\) at each iteration.
\end{itemize}
\end{frame}

\begin{frame}{}
\protect\hypertarget{section-15}{}
\textbf{Lawrence et al.~{[}2003{]}} suggest:

\begin{itemize}
\item
  the next point active set point can maximize the differential entropy
  score
  \(\Delta_j \stackrel {\bigtriangleup} = H[p(f_j)] - H[p^{new}(f_j)]\)
\item
  where \(H[p(f_j)]\) is the entropy of the Gaussian at site \(j \in R\)
  (which is a function of the variance at site \(j\) as the posterior is
  Gaussian, see eq. (A.20))
\item
  \(H[p^{new}(f_j)]\) is the entropy at this site once the observation
  at site \(j\) has been included.
\item
  Let the posterior variance of \(f_j\) before inclusion be \(v_j\).
\item
  As
  \(p(f_j|\pmb y_I, y_j) \propto p(f_j|\pmb y_I)N(y_j|f_j, \sigma^2)\)
  we have \((v_j^{new})^{-1} = v_j^{-1} + \sigma^{-2}\).
\end{itemize}
\end{frame}

\begin{frame}{}
\protect\hypertarget{section-16}{}
\begin{itemize}
\tightlist
\item
  Using the fact that the entropy of a Gaussian with variance \(v\) is
  \(\log(2\pi e v)/2\)
\end{itemize}

\[
\Delta_j = \frac 1 2 \log \Big(1 + \frac {v_j} {\sigma^2}\Big) \ \ \ \ \ (8.19) 
\]

\begin{itemize}
\item
  \(\Delta_j\) is a monotonic function of \(v_j\) so that it is
  maximized by choosing the site with the largest variance.
\item
  \textbf{Lawrence et al.~{[}2003{]}} call their method the informative
  IVM vector machine (IVM)
\end{itemize}
\end{frame}

\begin{frame}{}
\protect\hypertarget{section-17}{}
\begin{itemize}
\item
  Coded naively computing the variance at all sites in \(R\) cost
  \(\mathcal O(m^3 + (n - m)m^2)\) as we need to evaluate eq. (2.26) at
  each site
\item
  the matrix inversion of \(K_{mm} + \sigma_n^2 I\) can be done once in
  \(\mathcal O(m^3)\) then stored.
\item
  However, as we are incrementally growing the matrices \(K_{mm}\) and
  \(K_{m(n-m)}\) in fact the cost is \(\mathcal O(mn)\) per inclusion
\item
  leading to an overall complexity of \(\mathcal O(m^2n)\) when using a
  subset of size \(m\).
\end{itemize}
\end{frame}

\begin{frame}{}
\protect\hypertarget{section-18}{}
For example, once a site has been chosen for inclusion the matrix
\(K_{mm} + \sigma_n^2I\) is grown by including an extra row and column.

\begin{itemize}
\tightlist
\item
  The inverse of this expanded matrix can be found using eq. (A.12)
  although it would be better practice numerically to use a Cholesky
  decomposition approach as described in Lawrence et al.~{[}2003{]}.
\end{itemize}

\[
\begin{split}
& A= 
\begin{pmatrix}
P & Q\\
R & S
\end{pmatrix},\ \
A^{-1}= 
\begin{pmatrix}
\tilde P & \tilde Q\\
\tilde R & \tilde S
\end{pmatrix} \ \ \ \ \ (A.11)\\
& \begin{cases}
\tilde P = P^{-1} + P^{-1}QMRP^{-1}\\
\tilde Q=-P^{-1}QM\\
\tilde R = -MRP^{-1}\\
\tilde S = M
\end{cases}
M = (S-RP^{-1}Q)^{-1} \ \ \ \ (A. 12)
\end{split}
\]
\end{frame}

\begin{frame}{}
\protect\hypertarget{section-19}{}
\begin{itemize}
\item
  The scheme evaluates \(\Delta_j\) over all \(j \in R\) at each step to
  choose the inclusion site.
\item
  This makes sense when \(m\) is small, but as it gets larger it can
  make sense to select candidate inclusion sites from a subset of \(R\).
\item
  \textbf{Lawrence et al.~{[}2003{]}} call this the \textbf{randomized
  greedy selection method} and give further ideas on how to choose the
  subset.

  \begin{itemize}
  \tightlist
  \item
    The differential entropy score \(\Delta_j\) is not the only
    criterion that can be used for site selection.
  \item
    For example the information gain criterion
    \(KL(p^{new}(f_j)||p(f_j))\) can also be used.
  \end{itemize}
\item
  The use of greedy selection heuristics here is similar to the problem
  of active learning, see e.g.~\textbf{MacKay {[}1992c{]}}.
\end{itemize}
\end{frame}

\end{document}
